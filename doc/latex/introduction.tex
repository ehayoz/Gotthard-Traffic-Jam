Nowadays, traffic jams belongs to every day life. Because Switzerland is an important transit way to Southern Europe for traffic, the Gotthard tunnel becomes to an unavoidable bottleneck. The tunnel stretches out over 15 Kilometers from Göschenen, Kanton Uri to Airolo in the canton of Ticino. Over and over again it comes to long traffic jams, at peak times up to 20 Kilometers. Responsible for that are  on the one hand holidays in Switzerland or neighbor countries and on the other hand the lane reduction from two to one lane. Switzerland has been discussing for many years about this issue and building a second tunnel. The purpose of the second tunnel should be built for security reasons an not in order to increase the traffic flow. This is due to an election 20 years ago, called the Alpenschutzinitiative. 
In near future there will be a restauration for the tunnel. That's why the discussion about a second tunnel will go on again.

The authors have the goal to investigate the Gotthard tunnel on the north side in Göschenen. That's why a model has been created. By means of the number of cars passing through Erstfeld (a village 19 Kilometers distance from the tunnel), the length of congestion can be measured. The Nagel-Schreckenberg model will be employed for the simulation.


\subsection{Motivation}
As one of the authors is born in the canton of Uri he knows about the traffic issues at the Gotthard tunnel. This explains the personal motivation to investigate the phenomenon of traffic congestions.














