\textit{Abstract of Eric Hayoz \& Janick Zwyssig, Modelling the phenomenon of congestion at Gotthard -- A case study, 2014:}

This work describes the modelling of congestion on the stage between Erstfeld and the Gotthard tunnel in Göschenen, in the canton of Uri. In a first step a simulation based on the Nagel-Schreckenberg model was created. The model was supplemented by a second lane, a red-light and the ability to change the lane. A mapping from real data to the model allows to compare values. 
The main goal is to train the model that one can compare the measured congestion to a real traffic jam. For that the authors have implemented a congestion measurement and have processed datasets, provided by ASTRA (Amt für Strassen). A parameter setting that fits the reference data best was found. 

Finally the trained model was feed by two other, unused datasets to compare the results. Unfortunately, due to abnormalities of the datasets and some restrictions of the model there are big deviations to reference data. 