The authors created a model based on the Nagel-Schreckenberg model. It was extended to fit the situation at the
Gotthard-Strassentunnel Nordportal between Erstfeld and Göschenenen. Its purpose is to predict traffic congestions in
front of the tunnel.

The model was firstly trained with two datasets (24h and 48h), which were both provided by ASTRA.

The authors soon realised that there were some uncasualities in the congestion plot diagrams, which lead back to
measuring mistakes or inaccuracy of the dataset. They had to adjust these abnormalities in order to obtain appropriate output.

At the very end, the finished model had to prove its accuracy by predicting traffic congestions with two other datasets
(24h and 48h, also provided by ASTRA), which have not been used before.


\subsection{Outlook}
Both of the authors invested much time and effort in this project, and they really liked working on it. It is indisputable
that some misleading measurements from the datasets distorted the results, but the results...

Without a doubt systems of this nature will play an important role in the future.
